% Created 2024-06-28 Fri 00:34
% Intended LaTeX compiler: pdflatex
\documentclass[10pt]{article}
\usepackage[utf8]{inputenc}
\usepackage[T1]{fontenc}
\usepackage{graphicx}
\usepackage{longtable}
\usepackage{wrapfig}
\usepackage{rotating}
\usepackage[normalem]{ulem}
\usepackage{amsmath}
\usepackage{amssymb}
\usepackage{capt-of}
\usepackage{hyperref}
\usepackage[a4paper,left=1cm,right=1cm,top=1cm,bottom=1cm]{geometry}
\usepackage[american, english]{babel}
\usepackage{enumitem}
\usepackage{float}
\usepackage[sc]{mathpazo}
\usepackage{braket}
\usepackage{mathtools}
\linespread{1.05}
\renewcommand{\labelitemi}{$\rhd$}
\setlength\parindent{0pt}
\setlist[itemize]{leftmargin=*}
\setlist{nosep}
\newcommand{\invroot}[1]{\frac{1}{\sqrt{#1}}}
\author{Marcio Woitek}
\date{}
\title{Quantum Parallelism and Implementing Classical Gates}
\hypersetup{
 pdfauthor={Marcio Woitek},
 pdftitle={Quantum Parallelism and Implementing Classical Gates},
 pdfkeywords={},
 pdfsubject={},
 pdfcreator={Emacs 29.4 (Org mode 9.8)}, 
 pdflang={English}}
\begin{document}

\maketitle
\thispagestyle{empty}
\pagestyle{empty}
\section*{Problem 1}
\label{sec:orgc525722}
\textbf{Answer:}
\begin{itemize}
\item Applying \(U\) to the state of the form \(a\ket{00}+b\ket{11}\) yields the
super position \(b\ket{00}+a\ket{11}\).
\item Applying \(U\) to the uniform superposition \(\ket{\psi}\) of two qubits
will leave it unaltered.
\item The unitary matrix \(U\) will be \(\begin{psmallmatrix}
    0 & 0 & 0 & 1\\
    0 & 1 & 0 & 0\\
    0 & 0 & 1 & 0\\
    1 & 0 & 0 & 0
  \end{psmallmatrix}\).\\
\end{itemize}

\uline{Applying \(U\) to the state of the form \(a\ket{00}+b\ket{11}\) yields the
super position \(b\ket{00}+a\ket{11}\):}\\
According to the problem statement, the operator \(U\) satisfies the
following:
\begin{align}
  \begin{split}
    U\ket{00}&=\ket{11},\\
    U\ket{11}&=\ket{00},\\
    U\ket{01}&=\ket{01},\\
    U\ket{10}&=\ket{10}.
  \end{split}
\end{align}
Hence:
\begin{align}
  \begin{split}
    U(a\ket{00}+b\ket{11})&=aU\ket{00}+bU\ket{11}\\
    &=a\ket{11}+b\ket{00}\\
    &=b\ket{00}+a\ket{11}
  \end{split}
\end{align}
Therefore, this option is \textbf{correct}.\\

\uline{Applying \(U\) to the uniform superposition \(\ket{\psi}\) of two qubits
will leave it unaltered:}\\
Recall that the uniform superposition can be written as
\begin{equation}
\ket{\psi}=\frac{1}{2}(\ket{00}+\ket{01}+\ket{10}+\ket{11}).
\end{equation}
By applying \(U\) to this vector, we get
\begin{align}
  \begin{split}
    U\ket{\psi}&=\frac{1}{2}(U\ket{00}+U\ket{01}+U\ket{10}+U\ket{11})\\
    &=\frac{1}{2}(\ket{11}+\ket{01}+\ket{10}+\ket{00})\\
    &=\frac{1}{2}(\ket{00}+\ket{01}+\ket{10}+\ket{11})\\
    &=\ket{\psi}.
  \end{split}
\end{align}
Therefore, this option is \textbf{correct}.\\

Another \textbf{correct option} is the following:\\
The unitary matrix \(U\) will be
\begin{equation}
\begin{pmatrix}
  0 & 0 & 0 & 1\\
  0 & 1 & 0 & 0\\
  0 & 0 & 1 & 0\\
  1 & 0 & 0 & 0
\end{pmatrix}.
\end{equation}
This fact can be easily checked by multiplying this matrix by the column vectors
corresponding to the pure states. We won't present these calculations. Notice
that the only remaining option must be \textbf{wrong}, since it contradicts the one
above.

\newpage
\section*{Problem 2}
\label{sec:org50fab4f}
\textbf{Answer:}
\begin{itemize}
\item The multicontrol-X gate with two control qubits is also called the ``Toffoli''
gate.
\item Apply a multicontrolled X gate with \(b_1,b_2\) as the control gates
operating on the ancillary qubit \(b\). Note that multi-controlled X gate
will apply the X gate onto \(b\) only if \(b_1,b_2\) are both 1 or
otherwise leave the result unchanged.\\
\end{itemize}

Let's start from the truth table for the operation we wish to implement:
\begin{center}
\begin{tabular}{|c|c|c|c|c|}
\hline
\(b_1\) & \(b_2\) & \(b\) & \(b_1\wedge b_2\) & \(b\oplus(b_1\wedge b_2)\)\\
\hline
0 & 0 & 0 & 0 & 0\\
0 & 0 & 1 & 0 & 1\\
0 & 1 & 0 & 0 & 0\\
0 & 1 & 1 & 0 & 1\\
1 & 0 & 0 & 0 & 0\\
1 & 0 & 1 & 0 & 1\\
1 & 1 & 0 & 1 & 1\\
1 & 1 & 1 & 1 & 0\\
\hline
\end{tabular}
\end{center}
Next, denote the desired unitary operator by \(U\). We can convert our table
into a set of equations that \(U\) must satisfy:
\begin{align}
  \begin{split}
    U\ket{000}&=\ket{000},\\
    U\ket{001}&=\ket{001},\\
    U\ket{010}&=\ket{010},\\
    U\ket{011}&=\ket{011},\\
    U\ket{100}&=\ket{100},\\
    U\ket{101}&=\ket{101},\\
    U\ket{110}&=\ket{111},\\
    U\ket{111}&=\ket{110}.
  \end{split}
\end{align}
Notice we've used the fact that \(U\) leaves \(b_1\) and \(b_2\)
unchanged. Also notice that, in many cases, \(U\) acts like the identity
operator. In fact, there are only two cases in which \(U\) acts differently.
For the sake of clarity, we write the corresponding equations one more time:
\begin{align}
  \begin{split}
    U\ket{110}&=\ket{111},\\
    U\ket{111}&=\ket{110}.
  \end{split}
\end{align}
These equations show that \(U\) is a CNOT gate with two control qubits and one
target qubit. After all, the action of this operator is to negate \(b_3\), but
only when \(b_1=b_2=1\). Therefore, the option that talks about a
multicontrolled gate is \textbf{correct}. Moreover, the option that discusses a
different implementation must be \textbf{wrong}.\\
With the aid of the above equations, it's straightforward to determine the
matrix representation of \(U\):
\begin{equation}
U=
  \begin{pmatrix}
    \braket{000|U|000} & \braket{000|U|001} & \braket{000|U|010} & \braket{000|U|011} & \braket{000|U|100} & \braket{000|U|101} & \braket{000|U|110} & \braket{000|U|111}\\
    \braket{001|U|000} & \braket{001|U|001} & \braket{001|U|010} & \braket{001|U|011} & \braket{001|U|100} & \braket{001|U|101} & \braket{001|U|110} & \braket{001|U|111}\\
    \braket{010|U|000} & \braket{010|U|001} & \braket{010|U|010} & \braket{010|U|011} & \braket{010|U|100} & \braket{010|U|101} & \braket{010|U|110} & \braket{010|U|111}\\
    \braket{011|U|000} & \braket{011|U|001} & \braket{011|U|010} & \braket{011|U|011} & \braket{011|U|100} & \braket{011|U|101} & \braket{011|U|110} & \braket{011|U|111}\\
    \braket{100|U|000} & \braket{100|U|001} & \braket{100|U|010} & \braket{100|U|011} & \braket{100|U|100} & \braket{100|U|101} & \braket{100|U|110} & \braket{100|U|111}\\
    \braket{101|U|000} & \braket{101|U|001} & \braket{101|U|010} & \braket{101|U|011} & \braket{101|U|100} & \braket{101|U|101} & \braket{101|U|110} & \braket{101|U|111}\\
    \braket{110|U|000} & \braket{110|U|001} & \braket{110|U|010} & \braket{110|U|011} & \braket{110|U|100} & \braket{110|U|101} & \braket{110|U|110} & \braket{110|U|111}\\
    \braket{111|U|000} & \braket{111|U|001} & \braket{111|U|010} & \braket{111|U|011} & \braket{111|U|100} & \braket{111|U|101} & \braket{111|U|110} & \braket{111|U|111}
  \end{pmatrix}.
\end{equation}
We know how the operator \(U\) acts on the pure states. By using this
information, we can simplify the last expression to
\begin{equation}
U=
  \begin{pmatrix}
    \braket{000|000} & \braket{000|001} & \braket{000|010} & \braket{000|011} & \braket{000|100} & \braket{000|101} & \braket{000|111} & \braket{000|110}\\
    \braket{001|000} & \braket{001|001} & \braket{001|010} & \braket{001|011} & \braket{001|100} & \braket{001|101} & \braket{001|111} & \braket{001|110}\\
    \braket{010|000} & \braket{010|001} & \braket{010|010} & \braket{010|011} & \braket{010|100} & \braket{010|101} & \braket{010|111} & \braket{010|110}\\
    \braket{011|000} & \braket{011|001} & \braket{011|010} & \braket{011|011} & \braket{011|100} & \braket{011|101} & \braket{011|111} & \braket{011|110}\\
    \braket{100|000} & \braket{100|001} & \braket{100|010} & \braket{100|011} & \braket{100|100} & \braket{100|101} & \braket{100|111} & \braket{100|110}\\
    \braket{101|000} & \braket{101|001} & \braket{101|010} & \braket{101|011} & \braket{101|100} & \braket{101|101} & \braket{101|111} & \braket{101|110}\\
    \braket{110|000} & \braket{110|001} & \braket{110|010} & \braket{110|011} & \braket{110|100} & \braket{110|101} & \braket{110|111} & \braket{110|110}\\
    \braket{111|000} & \braket{111|001} & \braket{111|010} & \braket{111|011} & \braket{111|100} & \braket{111|101} & \braket{111|111} & \braket{111|110}
  \end{pmatrix}.
\end{equation}
This matrix still looks complicated, but most of its elements vanish. This is a
consequence of the fact that the inner products
\(\braket{b_1b_2b_3|b_{1}^{\prime}b_{2}^{\prime}b_{3}^{\prime}}\)
are non-zero only when \(b_i=b_{i}^{\prime},\:\forall i\). This allows us to
write \(U\) in the following way:
\begin{equation}
U=
  \begin{pmatrix}
    1 & 0 & 0 & 0 & 0 & 0 & 0 & 0\\
    0 & 1 & 0 & 0 & 0 & 0 & 0 & 0\\
    0 & 0 & 1 & 0 & 0 & 0 & 0 & 0\\
    0 & 0 & 0 & 1 & 0 & 0 & 0 & 0\\
    0 & 0 & 0 & 0 & 1 & 0 & 0 & 0\\
    0 & 0 & 0 & 0 & 0 & 1 & 0 & 0\\
    0 & 0 & 0 & 0 & 0 & 0 & 0 & 1\\
    0 & 0 & 0 & 0 & 0 & 0 & 1 & 0
  \end{pmatrix}.
\end{equation}
This operator is known as the \textbf{Toffoli or CCNOT gate}.\\
There's only one option remaining: it talks about ``checking \(b_1\) and \(b_2\)''.
The corresponding statement is \textbf{wrong}. In Quantum Computing, one doesn't simply
``check'' the value of a qubit. That would require a measurement, which affects
the circuit operation.
\section*{Problem 3}
\label{sec:org737f7ad}
\textbf{Answer:} This can be done by first applying a multi-controlled X (Toffoli) gate
on \(b_3\) with \(b_1,b_2\) as control qubits and then applying a controlled
X gate on \(b_2\) with \(b_1\) as the control qubit.\\
\textbf{Note:} I don't believe this option is actually right, but it is accepted as
correct.\\

Let's start from the truth table for the operation we wish to implement:
\begin{center}
\begin{tabular}{|c|c|c|c|c|}
\hline
\(b_1\) & \(b_2\) & \(b_3\) & \(b_1\oplus b_2\) & \(b_1\oplus b_2\oplus b_3\)\\
\hline
0 & 0 & 0 & 0 & 0\\
0 & 0 & 1 & 0 & 1\\
0 & 1 & 0 & 1 & 1\\
0 & 1 & 1 & 1 & 0\\
1 & 0 & 0 & 1 & 1\\
1 & 0 & 1 & 1 & 0\\
1 & 1 & 0 & 0 & 0\\
1 & 1 & 1 & 0 & 1\\
\hline
\end{tabular}
\end{center}
Next, denote the desired unitary operator by \(U\). We can convert our table
into a set of equations that \(U\) must satisfy:
\begin{align}
  \begin{split}
    U\ket{000}&=\ket{000},\\
    U\ket{001}&=\ket{001},\\
    U\ket{010}&=\ket{011},\\
    U\ket{011}&=\ket{010},\\
    U\ket{100}&=\ket{111},\\
    U\ket{101}&=\ket{110},\\
    U\ket{110}&=\ket{100},\\
    U\ket{111}&=\ket{101}.
  \end{split}
\end{align}
With the aid of the above equations, it's straightforward to determine the
matrix representation of \(U\):
\begin{equation}
U=
  \begin{pmatrix}
    \braket{000|U|000} & \braket{000|U|001} & \braket{000|U|010} & \braket{000|U|011} & \braket{000|U|100} & \braket{000|U|101} & \braket{000|U|110} & \braket{000|U|111}\\
    \braket{001|U|000} & \braket{001|U|001} & \braket{001|U|010} & \braket{001|U|011} & \braket{001|U|100} & \braket{001|U|101} & \braket{001|U|110} & \braket{001|U|111}\\
    \braket{010|U|000} & \braket{010|U|001} & \braket{010|U|010} & \braket{010|U|011} & \braket{010|U|100} & \braket{010|U|101} & \braket{010|U|110} & \braket{010|U|111}\\
    \braket{011|U|000} & \braket{011|U|001} & \braket{011|U|010} & \braket{011|U|011} & \braket{011|U|100} & \braket{011|U|101} & \braket{011|U|110} & \braket{011|U|111}\\
    \braket{100|U|000} & \braket{100|U|001} & \braket{100|U|010} & \braket{100|U|011} & \braket{100|U|100} & \braket{100|U|101} & \braket{100|U|110} & \braket{100|U|111}\\
    \braket{101|U|000} & \braket{101|U|001} & \braket{101|U|010} & \braket{101|U|011} & \braket{101|U|100} & \braket{101|U|101} & \braket{101|U|110} & \braket{101|U|111}\\
    \braket{110|U|000} & \braket{110|U|001} & \braket{110|U|010} & \braket{110|U|011} & \braket{110|U|100} & \braket{110|U|101} & \braket{110|U|110} & \braket{110|U|111}\\
    \braket{111|U|000} & \braket{111|U|001} & \braket{111|U|010} & \braket{111|U|011} & \braket{111|U|100} & \braket{111|U|101} & \braket{111|U|110} & \braket{111|U|111}
  \end{pmatrix}.
\end{equation}
We know how the operator \(U\) acts on the pure states. By using this
information, we can simplify the last expression to
\begin{equation}
U=
  \begin{pmatrix}
    \braket{000|000} & \braket{000|001} & \braket{000|011} & \braket{000|010} & \braket{000|111} & \braket{000|110} & \braket{000|100} & \braket{000|101}\\
    \braket{001|000} & \braket{001|001} & \braket{001|011} & \braket{001|010} & \braket{001|111} & \braket{001|110} & \braket{001|100} & \braket{001|101}\\
    \braket{010|000} & \braket{010|001} & \braket{010|011} & \braket{010|010} & \braket{010|111} & \braket{010|110} & \braket{010|100} & \braket{010|101}\\
    \braket{011|000} & \braket{011|001} & \braket{011|011} & \braket{011|010} & \braket{011|111} & \braket{011|110} & \braket{011|100} & \braket{011|101}\\
    \braket{100|000} & \braket{100|001} & \braket{100|011} & \braket{100|010} & \braket{100|111} & \braket{100|110} & \braket{100|100} & \braket{100|101}\\
    \braket{101|000} & \braket{101|001} & \braket{101|011} & \braket{101|010} & \braket{101|111} & \braket{101|110} & \braket{101|100} & \braket{101|101}\\
    \braket{110|000} & \braket{110|001} & \braket{110|011} & \braket{110|010} & \braket{110|111} & \braket{110|110} & \braket{110|100} & \braket{110|101}\\
    \braket{111|000} & \braket{111|001} & \braket{111|011} & \braket{111|010} & \braket{111|111} & \braket{111|110} & \braket{111|100} & \braket{111|101}
  \end{pmatrix}.
\end{equation}
This matrix still looks complicated, but most of its elements vanish. This is a
consequence of the fact that the inner products
\(\braket{b_1b_2b_3|b_{1}^{\prime}b_{2}^{\prime}b_{3}^{\prime}}\)
are non-zero only when \(b_i=b_{i}^{\prime},\:\forall i\). This allows us to
write \(U\) in the following way:
\begin{equation}
U=
  \begin{pmatrix}
    1 & 0 & 0 & 0 & 0 & 0 & 0 & 0\\
    0 & 1 & 0 & 0 & 0 & 0 & 0 & 0\\
    0 & 0 & 0 & 1 & 0 & 0 & 0 & 0\\
    0 & 0 & 1 & 0 & 0 & 0 & 0 & 0\\
    0 & 0 & 0 & 0 & 0 & 0 & 1 & 0\\
    0 & 0 & 0 & 0 & 0 & 0 & 0 & 1\\
    0 & 0 & 0 & 0 & 0 & 1 & 0 & 0\\
    0 & 0 & 0 & 0 & 1 & 0 & 0 & 0
  \end{pmatrix}.
\end{equation}
It's possible to show that this matrix is invertible. Its inverse is given by
\begin{equation}
U^{-1}=
  \begin{pmatrix}
    1 & 0 & 0 & 0 & 0 & 0 & 0 & 0\\
    0 & 1 & 0 & 0 & 0 & 0 & 0 & 0\\
    0 & 0 & 0 & 1 & 0 & 0 & 0 & 0\\
    0 & 0 & 1 & 0 & 0 & 0 & 0 & 0\\
    0 & 0 & 0 & 0 & 0 & 0 & 0 & 1\\
    0 & 0 & 0 & 0 & 0 & 0 & 1 & 0\\
    0 & 0 & 0 & 0 & 1 & 0 & 0 & 0\\
    0 & 0 & 0 & 0 & 0 & 1 & 0 & 0
  \end{pmatrix}.
\end{equation}
By comparing the last two expressions, one can conclude that \(U\) and
\(U^{-1}\) are \textbf{not} equal. This eliminates one of the options.\\
Next, we check whether \(U\) is unitary. Since we already know its inverse, we
only need to compute its adjoint. It's easy to show that
\begin{equation}
U^{\dagger}=
  \begin{pmatrix}
    1 & 0 & 0 & 0 & 0 & 0 & 0 & 0\\
    0 & 1 & 0 & 0 & 0 & 0 & 0 & 0\\
    0 & 0 & 0 & 1 & 0 & 0 & 0 & 0\\
    0 & 0 & 1 & 0 & 0 & 0 & 0 & 0\\
    0 & 0 & 0 & 0 & 0 & 0 & 0 & 1\\
    0 & 0 & 0 & 0 & 0 & 0 & 1 & 0\\
    0 & 0 & 0 & 0 & 1 & 0 & 0 & 0\\
    0 & 0 & 0 & 0 & 0 & 1 & 0 & 0
  \end{pmatrix}.
\end{equation}
Once again, compare the last two equations. Clearly, we have
\(U^{-1}=U^{\dagger}\). Therefore, this operator is unitary. This means we've
found a unitary transformation that implements the desired operation.
Then two more options can be eliminated. Specifically, we can eliminate
\begin{itemize}
\item the option saying that \(U\) is not unitary, and
\item the option saying that it's not possible to implement the desired operation.
\end{itemize}
A single option remains. That must be the right one. Indeed, it is accepted as
correct. But I believe it's actually wrong. This option suggests that \(U\)
can be factored as follows: \(U=U_2U_1\), where \(U_1\) denotes the Toffoli
gate, and \(U_2\) represents a controlled X gate with \(b_1\) as the control
and \(b_2\) as the target. Perhaps I'm wrong, but I think that the three-qubit
version of this operator should act according to
\begin{align}
  \begin{split}
    U_2\ket{000}&=\ket{000},\\
    U_2\ket{001}&=\ket{001},\\
    U_2\ket{010}&=\ket{010},\\
    U_2\ket{011}&=\ket{011},\\
    U_2\ket{100}&=\ket{110},\\
    U_2\ket{101}&=\ket{111},\\
    U_2\ket{110}&=\ket{100},\\
    U_2\ket{111}&=\ket{101}.
  \end{split}
\end{align}
These equations tell us that \(U_2\) negates \(b_2\) only when \(b_1=1\).
Moreover, in every case, this operator leaves the value of \(b_3\) unaltered.
It's straightforward to show that the corresponding matrix representation is
\begin{equation}
U_2=
  \begin{pmatrix}
    1 & 0 & 0 & 0 & 0 & 0 & 0 & 0\\
    0 & 1 & 0 & 0 & 0 & 0 & 0 & 0\\
    0 & 0 & 1 & 0 & 0 & 0 & 0 & 0\\
    0 & 0 & 0 & 1 & 0 & 0 & 0 & 0\\
    0 & 0 & 0 & 0 & 0 & 0 & 1 & 0\\
    0 & 0 & 0 & 0 & 0 & 0 & 0 & 1\\
    0 & 0 & 0 & 0 & 1 & 0 & 0 & 0\\
    0 & 0 & 0 & 0 & 0 & 1 & 0 & 0
  \end{pmatrix}.
\end{equation}
By using this result along with the formula for \(U_1\) from the previous
problem, we can compute the product \(U_2U_1\). Recall that this should be
equal to \(U\). However, that's not what we get. One can verify that
\begin{equation}
U-U_2U_1=
  \begin{pmatrix}
    0 & 0 & 0 & 0 & 0 & 0 & 0 & 0\\
    0 & 0 & 0 & 0 & 0 & 0 & 0 & 0\\
    0 & 0 & -1 & 1 & 0 & 0 & 0 & 0\\
    0 & 0 & 1 & -1 & 0 & 0 & 0 & 0\\
    0 & 0 & 0 & 0 & 0 & 0 & 1 & -1\\
    0 & 0 & 0 & 0 & 0 & 0 & -1 & 1\\
    0 & 0 & 0 & 0 & -1 & 1 & 0 & 0\\
    0 & 0 & 0 & 0 & 1 & -1 & 0 & 0
  \end{pmatrix}.
\end{equation}
This should be a matrix with all elements equal to 0. Therefore, I believe the
remaining option is also wrong. I don't discard the possibility that I'm the one
who is incorrect. However, it wouldn't be the first time I catch this kind of
mistake.
\section*{Problem 4}
\label{sec:org9143ead}
\textbf{Answer:} The ancillary qubit collapses to either a 1 or a 0 and the input
qubits \(b_1,\ldots,b_n\) collapse to a uniform combination of all those
inputs for which \(f(b_1,\ldots,b_n)\) matches the measured result.\\

In the solution to the next problem, there is an example with \(n=2\) and
\(f(b_1,b_2)=b_1\wedge b_2\).
\section*{Problem 5}
\label{sec:org7168ce3}
\textbf{Answer:}
\begin{itemize}
\item The measurement of \(\ket{b}\) yields \(\ket{0}\) with \(\frac{3}{4}\)
probability and \(\ket{1}\) with \(\frac{1}{4}\) probability.
\item If we measure \(b=0\) the input qubits collapse into the super position
\(\invroot{3}(\ket{00}+\ket{01}+\ket{10})\).
\item If we measure \(b=1\) the input qubits collapse into \(\ket{11}\).\\
\end{itemize}

The initial state can be expressed as
\begin{equation}
\ket{\psi}=\frac{1}{2}(\ket{000}+\ket{010}+\ket{100}+\ket{110}).
\end{equation}
By applying the Toffoli gate to this vector, we get
\begin{align}
  \begin{split}
    U\ket{\psi}&=\frac{1}{2}(U\ket{000}+U\ket{010}+U\ket{100}+U\ket{110})\\
    &=\frac{1}{2}(\ket{000}+\ket{010}+\ket{100}+\ket{111}).
  \end{split}
\end{align}
This equation makes it clear that, if we measure the ancillary qubit, the value
0 occurs with probability \(\frac{3}{4}\), and 1 occurs with probability
\(\frac{1}{4}\). Moreover, if the result of this measurement is 0, then
the state of the input qubits collapses to
\begin{equation*}
\invroot{3}(\ket{00}+\ket{01}+\ket{10}).
\end{equation*}
On the other hand, if we measure \(b=1\), then the input qubits can be only in
the state \(\ket{11}\). Therefore, the statement ``Measuring \(b\) has no
effect on the input qubits'' is \textbf{wrong}.
\end{document}
