% Created 2024-06-25 Tue 03:32
% Intended LaTeX compiler: pdflatex
\documentclass[11pt]{article}
\usepackage[utf8]{inputenc}
\usepackage[T1]{fontenc}
\usepackage{graphicx}
\usepackage{longtable}
\usepackage{wrapfig}
\usepackage{rotating}
\usepackage[normalem]{ulem}
\usepackage{amsmath}
\usepackage{amssymb}
\usepackage{capt-of}
\usepackage{hyperref}
\usepackage[a4paper,left=1cm,right=1cm,top=1cm,bottom=1cm]{geometry}
\usepackage[american, english]{babel}
\usepackage{enumitem}
\usepackage{float}
\usepackage[sc]{mathpazo}
\usepackage{braket}
\linespread{1.05}
\renewcommand{\labelitemi}{$\rhd$}
\setlength\parindent{0pt}
\setlist[itemize]{leftmargin=*}
\setlist{nosep}
\newcommand{\invroot}[1]{\frac{1}{\sqrt{#1}}}
\author{Marcio Woitek}
\date{}
\title{Single Qubit Quantum Gates}
\hypersetup{
 pdfauthor={Marcio Woitek},
 pdftitle={Single Qubit Quantum Gates},
 pdfkeywords={},
 pdfsubject={},
 pdfcreator={Emacs 29.3 (Org mode 9.8)}, 
 pdflang={English}}
\begin{document}

\maketitle
\thispagestyle{empty}
\pagestyle{empty}

\newcommand{\twoByTwo}[4]{%
  \begin{pmatrix}
    #1 & #2\\
    #3 & #4
  \end{pmatrix}%
}
\section*{Problem 1}
\label{sec:org8e4de30}
\textbf{Answer:} \(\ket{1}\)
\section*{Problem 2}
\label{sec:org0c047db}
\begin{itemize}
\item It is represented by the matrix \(\twoByTwo{1}{0}{0}{i}\) in the
computational basis \(\ket{0}\), \(\ket{1}\).
\item It is a unitary operator.
\item It is the same as a phase gate with phase \(\frac{\pi}{2}\).
\end{itemize}
\section*{Problem 3}
\label{sec:org4dfdc42}
\begin{itemize}
\item Measuring both states yields either outcome 0, 1 with equal probabilities.
\item If we applied a Hadamard gate then \(H\ket{\varphi}=\ket{0}\) and
\(H\ket{\psi}=\frac{1}{2}((1+i)\ket{0}+(1-i)\ket{1})\).
\item The two states can be distinguished upon the application of the Hadamard gate
and measurement. One of the states always yields \(\ket{0}\) while the other
yields \(\ket{0}\) and \(\ket{1}\) with equal probabilities.
\end{itemize}
\section*{Problem 4}
\label{sec:org165ec98}
\begin{itemize}
\item \(\twoByTwo{0}{1}{1}{0}\)
\item \(\twoByTwo{\cos(\theta)}{\sin(\theta)}{-\sin(\theta)}{\cos(\theta)}\)
\item \(\invroot{2}\twoByTwo{1}{1}{1}{-1}\)\\
\end{itemize}

The option \(\invroot{2}\twoByTwo{i}{-i}{1}{1}\) is also correct. However,
it's considered wrong.
\end{document}
