% Created 2024-06-24 Mon 23:12
% Intended LaTeX compiler: pdflatex
\documentclass[11pt]{article}
\usepackage[utf8]{inputenc}
\usepackage[T1]{fontenc}
\usepackage{graphicx}
\usepackage{longtable}
\usepackage{wrapfig}
\usepackage{rotating}
\usepackage[normalem]{ulem}
\usepackage{amsmath}
\usepackage{amssymb}
\usepackage{capt-of}
\usepackage{hyperref}
\usepackage[a4paper,left=1cm,right=1cm,top=1cm,bottom=1cm]{geometry}
\usepackage[american, english]{babel}
\usepackage{enumitem}
\usepackage{float}
\usepackage[sc]{mathpazo}
\usepackage{braket}
\linespread{1.05}
\renewcommand{\labelitemi}{$\rhd$}
\setlength\parindent{0pt}
\setlist[itemize]{leftmargin=*}
\setlist{nosep}
\newcommand{\invroot}[1]{\frac{1}{\sqrt{#1}}}
\newcommand{\sqnorm}[1]{\left|#1\right|^2}
\author{Marcio Woitek}
\date{}
\title{Qubits}
\hypersetup{
 pdfauthor={Marcio Woitek},
 pdftitle={Qubits},
 pdfkeywords={},
 pdfsubject={},
 pdfcreator={Emacs 29.3 (Org mode 9.8)}, 
 pdflang={English}}
\begin{document}

\maketitle
\thispagestyle{empty}
\pagestyle{empty}
\section*{Problem 1}
\label{sec:org1c82b41}
\begin{itemize}
\item Measuring it yields 0 and 1 with equal probabilities
\(\left(\frac{1}{2}\right)\).
\item Upon measurement the super position collapses to one of the pure states
\(\ket{0}\) or \(\ket{1}\).
\end{itemize}
\section*{Problem 2}
\label{sec:org712a166}
\begin{itemize}
\item \(\frac{\sqrt{3}}{2}\ket{0}+\frac{1}{2}\ket{1}\)
\item 0
\item 1
\end{itemize}
\section*{Problem 3}
\label{sec:org5405d4d}
The general form of \(\ket{-}\) is
\begin{equation}
\ket{-}=c_0\ket{0}+c_1\ket{1},
\end{equation}
where \(\sqnorm{c_0}+\sqnorm{c_1}=1\). Since we want \(\ket{+}\) and
\(\ket{-}\) to be orthogonal, the following must hold:
\begin{equation}
\braket{+|-}=0\qquad\Rightarrow\qquad\invroot{2}c_0+\invroot{2}c_1=0\qquad\Rightarrow\qquad c_0+c_1=0.
\end{equation}
Then we have \(c_1=-c_0\). Assuming these constants are real numbers, we can
write the other condition as follows:
\begin{equation}
2c_0^2=1\qquad\Rightarrow\qquad c_0=\invroot{2}.
\end{equation}
Hence:
\begin{equation}
\ket{-}=\invroot{2}\left(\ket{0}-\ket{1}\right).
\end{equation}
\end{document}
