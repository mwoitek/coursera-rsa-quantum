% Created 2024-06-26 Wed 21:24
% Intended LaTeX compiler: pdflatex
\documentclass[11pt]{article}
\usepackage[utf8]{inputenc}
\usepackage[T1]{fontenc}
\usepackage{graphicx}
\usepackage{longtable}
\usepackage{wrapfig}
\usepackage{rotating}
\usepackage[normalem]{ulem}
\usepackage{amsmath}
\usepackage{amssymb}
\usepackage{capt-of}
\usepackage{hyperref}
\usepackage[a4paper,left=1cm,right=1cm,top=1cm,bottom=1cm]{geometry}
\usepackage[american, english]{babel}
\usepackage{enumitem}
\usepackage{float}
\usepackage[sc]{mathpazo}
\usepackage{braket}
\linespread{1.05}
\renewcommand{\labelitemi}{$\rhd$}
\setlength\parindent{0pt}
\setlist[itemize]{leftmargin=*}
\setlist{nosep}
\newcommand{\invroot}[1]{\frac{1}{\sqrt{#1}}}
\author{Marcio Woitek}
\date{}
\title{Multiple Qubit Quantum Gates}
\hypersetup{
 pdfauthor={Marcio Woitek},
 pdftitle={Multiple Qubit Quantum Gates},
 pdfkeywords={},
 pdfsubject={},
 pdfcreator={Emacs 29.4 (Org mode 9.8)}, 
 pdflang={English}}
\begin{document}

\maketitle
\thispagestyle{empty}
\pagestyle{empty}
\section*{Problem 1}
\label{sec:org833287d}
\textbf{Answer:}
\begin{itemize}
\item Apply a Hadamard operation to the first qubit and a Pauli-X (quantum not) gate
to the second qubit.\\
\end{itemize}

\uline{Apply a Pauli-X gate to the first qubit and a Hadamard gate to the second qubit:}\\
The initial state is \(\ket{00}=\ket{0}\otimes\ket{0}\). This option suggests
applying the operator \(U=X\otimes H\) to this state vector. Let's check if
this works:
\begin{equation}
U\ket{00}=(X\otimes H)(\ket{0}\otimes\ket{0})=X\ket{0}\otimes H\ket{0}.
\end{equation}
The Pauli gate transforms \(\ket{0}\) as follows:
\begin{equation}
X\ket{0}=
  \begin{pmatrix}
    0 & 1\\
    1 & 0
  \end{pmatrix}
  \begin{pmatrix}
    1\\
    0
  \end{pmatrix}=
  \begin{pmatrix}
    0\\
    1
  \end{pmatrix}=\ket{1}.
\end{equation}
Moreover, we already know that the result of applying \(H\) to \(\ket{0}\) is
\begin{equation}
H\ket{0}=\invroot{2}(\ket{0}+\ket{1}).
\end{equation}
Hence:
\begin{equation}
U\ket{00}=\ket{1}\otimes\left[\invroot{2}(\ket{0}+\ket{1})\right]=\invroot{2}(\ket{10}+\ket{11}).
\end{equation}
This doesn't match our desired final state. Therefore, this option is \textbf{wrong}.\\

\uline{Apply a Hadamard operation to the first qubit and a Pauli-X (quantum not) gate
to the second qubit:}\\
The initial state is \(\ket{00}=\ket{0}\otimes\ket{0}\). This option suggests
applying the operator \(U=H\otimes X\) to this state vector. Let's check if
this works:
\begin{align}
  \begin{split}
    U\ket{00}&=(H\otimes X)(\ket{0}\otimes\ket{0})\\
    &=H\ket{0}\otimes X\ket{0}\\
    &=\left[\invroot{2}(\ket{0}+\ket{1})\right]\otimes\ket{1}\\
    &=\invroot{2}(\ket{01}+\ket{11}).
  \end{split}
\end{align}
This is exactly what we wanted. Therefore, this option is \textbf{correct}. Also notice
that this result eliminates the option ``There is no unitary transformation that
can achieve this.'' After all, we've just shown that \(U\) yields the desired
final state.\\

The remaining option cannot be right, since it requires copying the initial
quantum state.
\section*{Problem 2}
\label{sec:org4c53e5a}
\textbf{Answer:}
\begin{itemize}
\item Apply the operator \(H\otimes I\otimes H\) where \(H\) stands for the
single qubit Hadamard gate while \(I\) is the single qubit identity
operator.
\item Apply a Hadamard gate to the first and third qubits but leave the second qubit
unaltered.\\
\end{itemize}

\uline{Apply the operator \(H\otimes I\otimes H\) where \(H\) stands for the single
qubit Hadamard gate while \(I\) is the single qubit identity operator:}\\
In this case, the initial state is given by
\begin{equation}
\ket{\psi}=\invroot{8}(\ket{000}+\ket{001}+\ket{010}+\ket{011}+\ket{100}+\ket{101}+\ket{110}+\ket{111}).
\end{equation}
We need to apply \(U=H\otimes I\otimes H\) to this state vector. This means
computing the following vector:
\begin{equation}
U\ket{\psi}=\invroot{8}(U\ket{000}+U\ket{001}+U\ket{010}+U\ket{011}+U\ket{100}+U\ket{101}+U\ket{110}+U\ket{111}).
\end{equation}
For the sake of clarity, we're going to compute each of
\(U\ket{000},\ldots,U\ket{111}\) separately:
\begin{align}
  \begin{split}
    U\ket{000}&=(H\otimes I\otimes H)(\ket{0}\otimes\ket{0}\otimes\ket{0})\\
    &=H\ket{0}\otimes I\ket{0}\otimes H\ket{0}\\
    &=\left[\invroot{2}(\ket{0}+\ket{1})\right]\otimes\ket{0}\otimes\left[\invroot{2}(\ket{0}+\ket{1})\right]\\
    &=\left[\invroot{2}(\ket{00}+\ket{10})\right]\otimes\left[\invroot{2}(\ket{0}+\ket{1})\right]\\
    &=\frac{1}{2}(\ket{000}+\ket{100}+\ket{001}+\ket{101})
  \end{split}
\end{align}
\begin{align}
  \begin{split}
    U\ket{001}&=(H\otimes I\otimes H)(\ket{0}\otimes\ket{0}\otimes\ket{1})\\
    &=H\ket{0}\otimes I\ket{0}\otimes H\ket{1}\\
    &=\left[\invroot{2}(\ket{0}+\ket{1})\right]\otimes\ket{0}\otimes\left[\invroot{2}(\ket{0}-\ket{1})\right]\\
    &=\left[\invroot{2}(\ket{00}+\ket{10})\right]\otimes\left[\invroot{2}(\ket{0}-\ket{1})\right]\\
    &=\frac{1}{2}(\ket{000}+\ket{100}-\ket{001}-\ket{101})
  \end{split}
\end{align}
\begin{align}
  \begin{split}
    U\ket{010}&=(H\otimes I\otimes H)(\ket{0}\otimes\ket{1}\otimes\ket{0})\\
    &=H\ket{0}\otimes I\ket{1}\otimes H\ket{0}\\
    &=\left[\invroot{2}(\ket{0}+\ket{1})\right]\otimes\ket{1}\otimes\left[\invroot{2}(\ket{0}+\ket{1})\right]\\
    &=\left[\invroot{2}(\ket{01}+\ket{11})\right]\otimes\left[\invroot{2}(\ket{0}+\ket{1})\right]\\
    &=\frac{1}{2}(\ket{010}+\ket{110}+\ket{011}+\ket{111})
  \end{split}
\end{align}
\begin{align}
  \begin{split}
    U\ket{011}&=(H\otimes I\otimes H)(\ket{0}\otimes\ket{1}\otimes\ket{1})\\
    &=H\ket{0}\otimes I\ket{1}\otimes H\ket{1}\\
    &=\left[\invroot{2}(\ket{0}+\ket{1})\right]\otimes\ket{1}\otimes\left[\invroot{2}(\ket{0}-\ket{1})\right]\\
    &=\left[\invroot{2}(\ket{01}+\ket{11})\right]\otimes\left[\invroot{2}(\ket{0}-\ket{1})\right]\\
    &=\frac{1}{2}(\ket{010}+\ket{110}-\ket{011}-\ket{111})
  \end{split}
\end{align}
\begin{align}
  \begin{split}
    U\ket{100}&=(H\otimes I\otimes H)(\ket{1}\otimes\ket{0}\otimes\ket{0})\\
    &=H\ket{1}\otimes I\ket{0}\otimes H\ket{0}\\
    &=\left[\invroot{2}(\ket{0}-\ket{1})\right]\otimes\ket{0}\otimes\left[\invroot{2}(\ket{0}+\ket{1})\right]\\
    &=\left[\invroot{2}(\ket{00}-\ket{10})\right]\otimes\left[\invroot{2}(\ket{0}+\ket{1})\right]\\
    &=\frac{1}{2}(\ket{000}-\ket{100}+\ket{001}-\ket{101})
  \end{split}
\end{align}
\begin{align}
  \begin{split}
    U\ket{101}&=(H\otimes I\otimes H)(\ket{1}\otimes\ket{0}\otimes\ket{1})\\
    &=H\ket{1}\otimes I\ket{0}\otimes H\ket{1}\\
    &=\left[\invroot{2}(\ket{0}-\ket{1})\right]\otimes\ket{0}\otimes\left[\invroot{2}(\ket{0}-\ket{1})\right]\\
    &=\left[\invroot{2}(\ket{00}-\ket{10})\right]\otimes\left[\invroot{2}(\ket{0}-\ket{1})\right]\\
    &=\frac{1}{2}(\ket{000}-\ket{100}-\ket{001}+\ket{101})
  \end{split}
\end{align}
\begin{align}
  \begin{split}
    U\ket{110}&=(H\otimes I\otimes H)(\ket{1}\otimes\ket{1}\otimes\ket{0})\\
    &=H\ket{1}\otimes I\ket{1}\otimes H\ket{0}\\
    &=\left[\invroot{2}(\ket{0}-\ket{1})\right]\otimes\ket{1}\otimes\left[\invroot{2}(\ket{0}+\ket{1})\right]\\
    &=\left[\invroot{2}(\ket{01}-\ket{11})\right]\otimes\left[\invroot{2}(\ket{0}+\ket{1})\right]\\
    &=\frac{1}{2}(\ket{010}-\ket{110}+\ket{011}-\ket{111})
  \end{split}
\end{align}
\begin{align}
  \begin{split}
    U\ket{111}&=(H\otimes I\otimes H)(\ket{1}\otimes\ket{1}\otimes\ket{1})\\
    &=H\ket{1}\otimes I\ket{1}\otimes H\ket{1}\\
    &=\left[\invroot{2}(\ket{0}-\ket{1})\right]\otimes\ket{1}\otimes\left[\invroot{2}(\ket{0}-\ket{1})\right]\\
    &=\left[\invroot{2}(\ket{01}-\ket{11})\right]\otimes\left[\invroot{2}(\ket{0}-\ket{1})\right]\\
    &=\frac{1}{2}(\ket{010}-\ket{110}-\ket{011}+\ket{111})
  \end{split}
\end{align}
The next step is to add all these results. To make this calculation easier,
first we'll compute the sums of consecutive terms:
\begin{align}
  \begin{split}
    U\ket{000}+U\ket{001}&=\ket{000}+\ket{100}\\
    U\ket{010}+U\ket{011}&=\ket{010}+\ket{110}\\
    U\ket{100}+U\ket{101}&=\ket{000}-\ket{100}\\
    U\ket{110}+U\ket{111}&=\ket{010}-\ket{110}
  \end{split}
\end{align}
With these equations, now it's simple to determine the state \(U\ket{\psi}\):
\begin{equation}
U\ket{\psi}=\frac{1}{2\sqrt{2}}\cdot 2(\ket{000}+\ket{010})=\invroot{2}(\ket{000}+\ket{010}).
\end{equation}
This is the desired final state. Therefore, this option is \textbf{correct}. Notice
that we've also eliminated the option ``There is no way to do this with a unitary
operation.''\\

\uline{Apply a Hadamard gate to the first and third qubits but leave the second qubit
unaltered:}\\
This option says the same as the one we've just shown is right. Therefore, this
option is also \textbf{correct}.\\

The remaining option is just silly. It's obviously \textbf{wrong}. We cannot simply
``zap components''.
\section*{Problem 3}
\label{sec:org8c4aac0}
\textbf{Answer:} Apply a controlled-X (quantum not) operation to the second qubit with
the first qubit as the control qubit.\\

\uline{Justification:}\\
The initial state is
\begin{equation}
\ket{\psi}=\invroot{2}(\ket{00}+\ket{10})=\invroot{2}
  \begin{pmatrix}
    1\\
    0\\
    1\\
    0
  \end{pmatrix}.
\end{equation}
The matrix representation of the operation described above can be written as
\begin{equation}
U=\begin{pmatrix}
    1 & 0 & 0 & 0\\
    0 & 1 & 0 & 0\\
    0 & 0 & 0 & 1\\
    0 & 0 & 1 & 0
  \end{pmatrix}.
\end{equation}
Then the transformed state can be computed as follows:
\begin{equation}
U\ket{\psi}=\invroot{2}
  \begin{pmatrix}
    1 & 0 & 0 & 0\\
    0 & 1 & 0 & 0\\
    0 & 0 & 0 & 1\\
    0 & 0 & 1 & 0
  \end{pmatrix}
  \begin{pmatrix}
    1\\
    0\\
    1\\
    0
  \end{pmatrix}=\invroot{2}
  \begin{pmatrix}
    1\\
    0\\
    0\\
    1
  \end{pmatrix}=\invroot{2}(\ket{00}+\ket{11}).
\end{equation}
This is exactly what we wanted. Therefore, this is the correct option.
\section*{Problem 4}
\label{sec:org147b1f5}
\textbf{Answer:} \(\invroot{2}(\ket{00}+i\ket{11})\)\\

First, recall that the phase gate with phase \(\frac{\pi}{2}\) is
\begin{equation}
P\left(\frac{\pi}{2}\right)=
  \begin{pmatrix}
    1 & 0\\
    0 & i
  \end{pmatrix}.
\end{equation}
Then the controlled gate described in the problem statement can be expressed as
\begin{equation}
U=\begin{pmatrix}
    1 & 0 & 0 & 0\\
    0 & 1 & 0 & 0\\
    0 & 0 & 1 & 0\\
    0 & 0 & 0 & i
  \end{pmatrix}.
\end{equation}
To obtain the desired result, all we need to do is to apply this operator to the
state vector \(\ket{\psi}=\invroot{2}(\ket{00}+\ket{11})\):
\begin{equation}
U\ket{\psi}=\invroot{2}
  \begin{pmatrix}
    1 & 0 & 0 & 0\\
    0 & 1 & 0 & 0\\
    0 & 0 & 1 & 0\\
    0 & 0 & 0 & i
  \end{pmatrix}
  \begin{pmatrix}
    1\\
    0\\
    0\\
    1
  \end{pmatrix}=\invroot{2}
  \begin{pmatrix}
    1\\
    0\\
    0\\
    i
  \end{pmatrix}=\invroot{2}(\ket{00}+i\ket{11}).
\end{equation}
\end{document}
