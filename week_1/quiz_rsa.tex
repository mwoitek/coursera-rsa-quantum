% Created 2024-06-20 Thu 23:38
% Intended LaTeX compiler: pdflatex
\documentclass[11pt]{article}
\usepackage[utf8]{inputenc}
\usepackage[T1]{fontenc}
\usepackage{graphicx}
\usepackage{longtable}
\usepackage{wrapfig}
\usepackage{rotating}
\usepackage[normalem]{ulem}
\usepackage{amsmath}
\usepackage{amssymb}
\usepackage{capt-of}
\usepackage{hyperref}
\usepackage[a4paper,left=1cm,right=1cm,top=1cm,bottom=1cm]{geometry}
\usepackage[american]{babel}
\usepackage{enumitem}
\usepackage{float}
\usepackage[sc]{mathpazo}
\linespread{1.05}
\renewcommand{\labelitemi}{$\rhd$}
\setlength\parindent{0pt}
\setlist[itemize]{leftmargin=*}
\setlist{nosep}
\newcommand{\Mod}[1]{\:\mathrm{mod}\:#1}
\author{Marcio Woitek}
\date{}
\title{Quiz on RSA}
\hypersetup{
 pdfauthor={Marcio Woitek},
 pdftitle={Quiz on RSA},
 pdfkeywords={},
 pdfsubject={},
 pdfcreator={Emacs 29.3 (Org mode 9.6.24)}, 
 pdflang={English}}
\begin{document}

\maketitle
\thispagestyle{empty}
\pagestyle{empty}

\section*{Problem 1}
\label{sec:org94ffc08}
\textbf{Answer: 20}

\begin{equation}
\varphi(33) = \varphi(3\cdot 11) = (3-1)(11-1) = 20.
\end{equation}

\section*{Problem 2}
\label{sec:orgd5e9c79}
\textbf{Answer: \(p-1\)}

\section*{Problem 3}
\label{sec:org41b959e}
\textbf{Answer: 4}\\[0pt]

When the consider the integers \(k\) in the range \(1\leq k\leq 12\), there are only 4
integers that are relatively prime to 12: 1, 5, 7, 11. Hence:
\begin{equation}
\varphi(12)=4.
\end{equation}

\section*{Problem 4}
\label{sec:org30bf22b}
\textbf{Answer: 1}\\[0pt]

First notice that \(77 = 7 \cdot 11\). Since 7 and 11 are prime numbers, the value
of the totient function for 77 is
\begin{equation}
\varphi(77) = (7 - 1) (11 - 1) = 60.
\end{equation}
This result allows us to write
\begin{equation*}
10^{60} \Mod 77 = 10^{\varphi(77)} \Mod 77.
\end{equation*}
Next, we use the fact that \(10 = 2 \cdot 5\). This expression makes it clear that
10 and 77 don't have prime factors in common. In other words, these numbers are
relatively prime. Then we can apply Euler's totient theorem, which yields
\begin{equation}
10^{60} \Mod 77 = 1.
\end{equation}

\section*{Problem 5}
\label{sec:org57326ed}
\textbf{Answer: 1}\\[0pt]

To determine the last digit of \(a^{4}\), we need to compute \(a^{4}\Mod 10\). To
compute this value, first notice that
\begin{equation}
\varphi(10)=\varphi(2\cdot 5)=(2-1)(5-1)=4.
\end{equation}
We'll also use the fact that \(a\) and 10 are relatively prime. This has to be
true, since by hypothesis \(a\) is not divisible by 2 or 5. This means 2 and 5 are
not prime factors of \(a\). But they are the only prime factors of 10. Then \(a\)
and 10 don't have prime factors in common, i.e., they're relatively prime.\\[0pt]
With the aid of Euler's theorem, we obtain the following for the last digit of
\(a^{4}\):
\begin{equation}
a^{4}\Mod 10=a^{\varphi(10)}\Mod 10=1.
\end{equation}

\section*{Problem 6}
\label{sec:org22a2167}
\begin{itemize}
\item \(\varphi(n)\): \(\varphi(n)=\varphi(33)=20\) [see Problem 1]
\item \(e^{\varphi(n)}\Mod n\): \(e\) and \(n\) are relatively prime, which means this
value equals 1 (by Euler's theorem).
\item \(\mathrm{GCD}(e,\varphi(n))\): \(\mathrm{GCD}(e,\varphi(n))=\mathrm{GCD}(7,20)=1\)
\item \((d,k)\) such that \(de-k\varphi(n)=1\): We need to solve the equation
\(7d-20k=1\). We could do that using the extended Euclidean algorithm. But it's
easy to guess a solution: \((d,k)=(3,1)\).
\item Public Key: \((e,n)=(7,33)\)
\item Private Key: \((d,n)=(3,33)\)
\end{itemize}

\section*{Problem 7}
\label{sec:orgc9dabde}
\begin{itemize}
\item The prime factors of \(n\).
\item The Euler totient function \(\varphi(n)\).
\end{itemize}
\end{document}
