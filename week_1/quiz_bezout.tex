% Created 2024-06-21 Fri 22:19
% Intended LaTeX compiler: pdflatex
\documentclass[11pt]{article}
\usepackage[utf8]{inputenc}
\usepackage[T1]{fontenc}
\usepackage{graphicx}
\usepackage{longtable}
\usepackage{wrapfig}
\usepackage{rotating}
\usepackage[normalem]{ulem}
\usepackage{amsmath}
\usepackage{amssymb}
\usepackage{capt-of}
\usepackage{hyperref}
\usepackage[a4paper,left=1cm,right=1cm,top=1cm,bottom=1cm]{geometry}
\usepackage[american]{babel}
\usepackage{enumitem}
\usepackage{float}
\usepackage[sc]{mathpazo}
\linespread{1.05}
\renewcommand{\labelitemi}{$\rhd$}
\setlength\parindent{0pt}
\setlist[itemize]{leftmargin=*}
\setlist{nosep}
\newcommand{\Mod}[1]{\:\mathrm{mod}\:#1}
\author{Marcio Woitek}
\date{}
\title{Bezout Coefficients}
\hypersetup{
 pdfauthor={Marcio Woitek},
 pdftitle={Bezout Coefficients},
 pdfkeywords={},
 pdfsubject={},
 pdfcreator={Emacs 29.3 (Org mode 9.6.24)}, 
 pdflang={English}}
\begin{document}

\maketitle
\thispagestyle{empty}
\pagestyle{empty}

\section*{Problem 1}
\label{sec:org425f046}
\textbf{Answer: \((7,-4)\)}\\[0pt]

When we apply the extended Euclidean algorithm with \(m=19\) and \(n=11\), this is
what we get:
\begin{center}
\begin{tabular}{|c|c|c|c|c|c|c|c|}
\hline
\(m\) & \(n\) & \(q\) & \(r\) & \(s\) & \(t\) & \(\hat{s}\) & \(\hat{t}\)\\[0pt]
\hline
19 & 11 & 1 & 8 & 1 & 0 & 0 & 1\\[0pt]
11 & 8 & 1 & 3 & 0 & 1 & 1 & -1\\[0pt]
8 & 3 & 2 & 2 & 1 & -1 & -1 & 2\\[0pt]
3 & 2 & 1 & 1 & -1 & 2 & 3 & -5\\[0pt]
2 & 1 & 2 & 0 & 3 & -5 & -4 & 7\\[0pt]
\hline
\end{tabular}
\end{center}
The desired result is in the last two columns of the last row. Specifically, the
coefficient corresponding to \(m\) is the final value of \(\hat{s}\), and the
coefficient associated with \(n\) is the last \(\hat{t}\). Therefore, \(t=-4\) and
\(s=7\). This has to be true, since \(19\cdot(-4)+11\cdot 7=-76+77=1\).

\section*{Problem 2}
\label{sec:org059245a}
\textbf{Answer: No, any number of the form \(24s+32t\) where \(s\), \(t\) are integers must
be divisible by 8.}\\[0pt]

Bob is going to receive \(s\) coins from Alice. This amounts to \(24s\). To make
this exchange work, Bob has to give Alice \(t\) coins. In this case, Bob is losing
money, which means the total is \(-32t\). Since the goal is to give Bob 4 cents,
the following equation must hold:
\begin{equation}
24s-32t=4.
\end{equation}
Next, denote the LHS of this equation by \(T\). We also introduce \(t^{\prime}=-t\).
Using these definitions, we can write
\begin{equation}
T=24s+32t^{\prime}=8(3s+4t^{\prime}).
\end{equation}
The last expression makes it clear that \(T\) is divisible by 8. Since 4 doesn't
have this property, it's impossible to satisfy \(T=4\).

\section*{Problem 3}
\label{sec:orgcf0e9c6}
\textbf{Answer: No such integer since \(15k\Mod 21\) must be divisible by 3 for all \(k\).}\\[0pt]

Assume it's possible to satisfy the equation we were given. In this case, we can
write \(15k\) as follows:
\begin{equation}
15k=21q+1,
\end{equation}
where \(q\) is some unknown integer. \(15k\) is clearly divisible by 3. Then the RHS
of the above equation must also be divisible by 3. We can express this fact
through the following equation:
\begin{equation}
(21q+1)\Mod 3=0.
\end{equation}
Notice that 21 is also divisible by 3. This allows us to write
\begin{equation}
(21q+1)\Mod 3=[(21q)\Mod 3+1\Mod 3]\Mod 3=1.
\end{equation}
Then we've just shown that \(1=0\). Since this is absurd, our assumption that the
original problem has a solution must be wrong.

\section*{Problem 4}
\label{sec:orgf904936}
\textbf{Answer: \(s\leftarrow s-qs^{\prime}\), \(t\leftarrow t-qt^{\prime}\)}\\[0pt]

To avoid confusion, let's denote the original \(m\), \(n\), \(q\) and \(r\) by \(m_0\),
\(n_0\), \(q_0\) and \(r_0\). Next, imagine we're applying the extended Euclidean
algorithm with \(m=m_0\) and \(n=n_0\). The first two steps are represented below.
\begin{center}
\begin{tabular}{|c|c|c|c|c|c|c|c|}
\hline
\(m\) & \(n\) & \(q\) & \(r\) & \(s\) & \(t\) & \(\hat{s}\) & \(\hat{t}\)\\[0pt]
\hline
\(m_0\) & \(n_0\) & \(q_0\) & \(r_0\) & 1 & 0 & 0 & 1\\[0pt]
\(n_0\) & \(r_0\) & \(n_0//r_0\) & \(n_0\%r_0\) & 0 & 1 & 1 & \(-q_0\)\\[0pt]
\hline
\end{tabular}
\end{center}
Notice that, starting from the second row, we're using the extended Euclidean
algorithm with \(m=n_0\) and \(n=r_0=m_0\Mod n_0\). This is the exact same situation
the problem statement talks about. This table also shows us that the
coefficients associated with \(r_0\) are \(\hat{s}\) and \(\hat{t}\). As we know,
these coefficients are updated by using the following rules:
\begin{align}
\hat{s}&\leftarrow s-q\hat{s},\\
\hat{t}&\leftarrow t-q\hat{t}.
\end{align}
Inspecting the first row of our table, we see that \(\hat{s}\) and \(\hat{t}\) are
also related to \(n_0\). In the problem statement, these coefficients are denoted
by \(s^{\prime}\) and \(t^{\prime}\). Therefore, the rules for updating the
coefficients related to \(r_0\) are
\begin{align}
\hat{s}&\leftarrow s-qs^{\prime},\\
\hat{t}&\leftarrow t-qt^{\prime}.
\end{align}
Our notation doesn't match the one used in the options, but it has the advantage
of not being confusing.
\end{document}
