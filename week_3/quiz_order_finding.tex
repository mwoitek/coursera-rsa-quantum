% Created 2024-06-28 Fri 21:45
% Intended LaTeX compiler: pdflatex
\documentclass[11pt]{article}
\usepackage[utf8]{inputenc}
\usepackage[T1]{fontenc}
\usepackage{graphicx}
\usepackage{longtable}
\usepackage{wrapfig}
\usepackage{rotating}
\usepackage[normalem]{ulem}
\usepackage{amsmath}
\usepackage{amssymb}
\usepackage{capt-of}
\usepackage{hyperref}
\usepackage[a4paper,left=1cm,right=1cm,top=1cm,bottom=1cm]{geometry}
\usepackage[american, english]{babel}
\usepackage{enumitem}
\usepackage{float}
\usepackage[sc]{mathpazo}
\linespread{1.05}
\renewcommand{\labelitemi}{$\rhd$}
\setlength\parindent{0pt}
\setlist[itemize]{leftmargin=*}
\setlist{nosep}
\newcommand{\Mod}{\mathrm{mod}\:}
\date{}
\title{Order Finding and Factoring}
\hypersetup{
 pdfauthor={Marcio Woitek},
 pdftitle={Order Finding and Factoring},
 pdfkeywords={},
 pdfsubject={},
 pdfcreator={Emacs 29.4 (Org mode 9.8)}, 
 pdflang={English}}
\begin{document}

\thispagestyle{empty}
\pagestyle{empty}
\section*{Problem 1}
\label{sec:org8d0e01e}

\textbf{Answer:} 4\\

This value was computed in one of the lectures. The order is \(r=4\), since
\(7^4\:\Mod 15=1\).
\section*{Problem 2}
\label{sec:org9450469}

\textbf{Answer:}
\begin{itemize}
\item \(7^{20}\:\Mod 55=1\)
\item \(7^{10}\:\Mod 55\neq 1\)
\item \((7^{10}+1)\:\Mod 55\neq 0\), and it has a common factor with 55.
\item \((7^{10}-1)\:\Mod 55\) has a common factor with 55 since
\((7^{10}+1)\:\Mod 55\neq 0\).\\
\end{itemize}

Since \(r=20\) is the order, the following must hold:
\begin{equation}
7^{20}\:\Mod 55=1.
\end{equation}
Next, consider the options related to \(7^{10}\:\Mod 55\). It's
straightforward to show that
\begin{equation}
7^{10}\:\Mod 55=34.
\end{equation}
Clearly, we have \(7^{10}\:\Mod 55\neq 1\). Moreover, 34 has no factor in
common with 55. After all, the corresponding prime factorizations are
\(34=2\cdot 17\) and \(55=5\cdot 11\).\\
With the aid of our last result, we can analyze the remaining options. First
notice that
\begin{align}
  \begin{split}
    \left(7^{10}+1\right)\Mod 55&=35,\\
    \left(7^{10}-1\right)\Mod 55&=33.
  \end{split}
\end{align}
The first equation tells us that 55 does not divide \(7^{10}+1\). In this
case, these numbers have a common factor given by
\begin{align}
  \begin{split}
    d&=\gcd\left(7^{10}+1,55\right)\\
    &=\gcd\left(55,\left(7^{10}+1\right)\Mod 55\right)\\
    &=\gcd(55,35)\\
    &=5.
  \end{split}
\end{align}
This result allows us to find another non-trivial factor of \(N=55\):
\begin{equation}
\frac{N}{d}=\frac{55}{5}=11.
\end{equation}
This is one of the prime factors of 33. Therefore, \((7^{10}-1)\:\Mod 55\) has a
common factor with 55.\\
The last remaining option is \textbf{wrong}. \((a^{r/2}-1)\:\Mod 55\) has a common
factor with 55 only when this number does not divide \(a^{r/2}+1\).
\section*{Problem 3}
\label{sec:orgdc2146a}

\textbf{Answer:}
\begin{itemize}
\item \(p\)
\item \(q\)\\
\end{itemize}

The problem statement describes the case in which \(n\) and \(a^{r/2}+1\)
have a non-trivial common factor. This factor is given by \(\gcd(a^{r/2}+1,n)\).
Since \(n\) has only two factors, \(p\) and \(q\), these are the only
possible results for the GCD.
\section*{Problem 4}
\label{sec:orgb1ad0da}

\textbf{Answer:} \(O(m^3)\): \(2m\) multiplications of \(m\) bit numbers and
\(2m\) modulo operations.\\

For every bit in the exponent \(k\), we need to perform at most 2
multiplication + modulo operations. In total, we perform at most \(2m\) such
operations. Since the time cost of a single operation is \(O(m^2)\), the
complexity of modular exponentiation is \(O(m^3)\).
\section*{Problem 5}
\label{sec:org2f9fe44}

\textbf{Answer:}
\begin{itemize}
\item Modular exponentiation is about computing \(a^k\:\Mod n\) once for a given
\(k\). However, order finding repeatedly needs to compute \(a^k\:\Mod n\)
for various \(k\) until the result is 1.
\item Repeated squaring yields \(a\:\Mod n,a^2\:\Mod n,a^4\:\Mod n,\ldots ,a^{2^k}\:\Mod n\).
However the actual order \(r\) such that \(a^r\:\Mod n=1\) need not be a
power of 2.
\end{itemize}
\end{document}
